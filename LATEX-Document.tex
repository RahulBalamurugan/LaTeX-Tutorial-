\documentclass[12pt]{article}

\usepackage{times}
\usepackage{hyperref}
\usepackage[margin=1in]{geometry}
\usepackage{graphicx}
\usepackage{float}
\usepackage{caption}
\usepackage[backend=biber,style=numeric]{biblatex}

\addbibresource{references.bib}

\title{ My First \LaTeX\ Report}
\author{Rahul Balamurugan \\ \textit{\href{mailto:leorahultn@gmail.com}{leorahultn@gmail.com}}}
\date{August 31, 2024}

\begin{document}

\maketitle

\tableofcontents
\newpage

\section{Introduction}

The report is a basic documentation of the various platforms that we have explored and played with. We are going to explore \LaTeX , Mathematica, R, Python, and Julia.

\newpage

\section{\LaTeX}

This section discusses \LaTeX\ and includes screenshots of various computational tools. \LaTeX is a document preparation system mainly used for high-quality typesetting.

\begin{enumerate}

\item Downloading \LaTeX\ 

\begin{figure}[H]
    \centering
    \includegraphics[width=0.5\textwidth]{Download of LaTeX.png}
    \caption{The path information of the \LaTeX\ download}
    \label{fig:latex_download}
\end{figure}

\begin{itemize}
    \item \textbf{For macOS}: Install MacTeX from the following link: \href{https://tug.org/mactex/mactex-download.html}{MacTeX Download Page}.
    \item \textbf{For Windows}: Install MiKTeX from the following link: \href{https://miktex.org/download}{MiKTeX Download Page}.
\end{itemize}

\newpage
\item Creating \LaTeX\ Document

\begin{figure}[H]
    \centering
    \includegraphics[width=0.5\textwidth]{Document_1.png}
    \caption{The \LaTeX\ Document 1.png image showing the initial commands.}
    \label{fig:latex_commands}
\end{figure}

The image above showcases a \LaTeX\ document created with a title, author, and date. It includes an itemized list discussing complex systems and uses several packages to enhance functionality and formatting. The references are added to the document using the \texttt{\textbackslash cite} function, and there is a hyperlink where the email ID is reflected.

\item Importing Image into \LaTeX

\begin{figure}[H]
    \centering
    \includegraphics[width=0.5\textwidth]{Document2.png}
    \caption{The LaTeX Document 2.png shows how to import an image into LaTeX.}
    \label{fig:latex_import_image}
\end{figure}

The image showcases the command for alignment and adding a caption to the imported image along with the size in terms of width. It's an image of a flock of birds that represents how I view complex systems.

\newpage

\item How does the Output look

\begin{figure}[H]
    \centering
    \includegraphics[width=0.5\textwidth]{Page1.png}
    \caption{The above image shows the look of the overall output discussed so far.}
    \label{fig:latex_output}
\end{figure}

\item Mathematics and Tables using \LaTeX

\begin{figure}[H]
    \centering
    \includegraphics[width=0.5\textwidth]{part2latex.png}
    \caption{The above image showcases the second half of the \LaTeX\ document created, which includes math and tables.}
    \label{fig:latex_math_tables}
\end{figure}

\newpage 

The equation environment centers the equation on its own line and automatically numbers it. This helps in presenting the equation clearly and allows you to refer to it easily later in your document. The fractions, summations, and labeling are included accordingly. The table values are assigned using the begin command which initiates the table environment as shown above. 

\item Reference using \LaTeX (Output)

As shown in the previous image there is a citation of Dr. Hiroki Sayama's book.

\begin{figure}[H]
    \centering
    \includegraphics[width=0.5\textwidth]{ref.png}
    \caption{The image shows the reference page.}
    \label{fig:latex_reference}
\end{figure}

This is an outcome of the cite function used beside the content which was cited along with the BibTeX reference used by utilizing the bibliography function.

\end{enumerate}

\newpage

\section{Python (Anaconda)}

This section discusses the Python Programming Language in Anaconda's Jupyter environment and includes screenshots.

\begin{enumerate}

\item Downloading Anaconda Navigator and Jupyter Notebook \href{https://www.anaconda.com/products/navigator}{Anaconda Navigator Download page}.

\begin{figure}[H]
    \centering
    \includegraphics[width=0.5\textwidth]{download1.png}
    \caption{Anaconda Navigator Home Page}
    \label{fig:anaconda_home}
\end{figure}

\begin{figure}[H]
    \centering
    \includegraphics[width=0.5\textwidth]{Pythona.png}
    \caption{Jupyter Notebook home page}
    \label{fig:jupyter_home}
\end{figure}

\newpage 

\item Python Initial Coding 

\begin{figure}[H]
    \centering
    \includegraphics[width=0.4\textwidth]{python1.png}
    \caption{Network Analysis of Economic Sectors using Python}
    \label{fig:python_network_analysis}
\end{figure}

This Python code creates a directed graph of economic sectors by providing weights to it and visualizing the connections.

\begin{figure}[H]
    \centering
    \includegraphics[width=0.5\textwidth]{Python3.png}
    \caption{Agent-Based Modeling of Market Dynamics}
    \label{fig:python_agent_based_modeling}
\end{figure}

\newpage 

This code simulates a simple market where buyers and sellers trade on prices that are generated randomly over 100 days in a dynamic manner; and it has been plotted, also has used various loops utilizing Python in the Jupyter Notebook. 

\section{Mathematica}
\begin{itemize}
    
\item This section explains Mathematica which is a computational software used for various purposes in many fields. 
 \item You can Download Mathematica from \href{https://wolfram.com/siteinfo/}{Wolfram for Mathematica}.

\end{itemize}

\begin{figure}[H]
    \centering
    \includegraphics[width=0.5\textwidth]{Mathematica1.png}
    \caption{Shows the initial function used in Mathematica}
    \label{fig:mathematica_function}
\end{figure}

The above diagram shows the Graphplot function used in the Petersen graph and its output.

\item Basic Mathematics 

\begin{figure}[H]
    \centering
    \includegraphics[width=0.5\textwidth]{Mathematicab.png}
    \caption{Shows mathematical functions used in Mathematica}
    \label{fig:mathematica_math_functions}
\end{figure}

The above image shows the various mathematical functions that can be performed in Mathematica Software such as basic arithmetic, integration, matrix and more!

\item Science using Mathematica

\begin{figure}[H]
    \centering
    \includegraphics[width=0.5\textwidth]{Science1.png}
    \caption{Shows how Mathematica can be utilized in the field of Chemistry} 
    \label{fig:mathematica_science}
\end{figure}

The command used in Mathematica showcased the molecular structure of Benzene and 3D plotting of Caffeine's molecular structure. 

\newpage
  
\item Can we also Integrate Audio? The answer is YES! :)
  
Here we can see how it's done
  
\begin{figure}[H]
    \centering
    \includegraphics[width=0.5\textwidth]{AudioM.png}
    \caption{How Audio can be integrated with Mathematica and analyzed} 
    \label{fig:mathematica_audio}
\end{figure}

Import the audio file into Mathematica, use the Audio Plot function to visualize the sound waves of the audio, then use sample rate and length functions to check the frequency and length of the audio. 

\end{enumerate}

\newpage

\section{R Programming (Rstudio)}

R Programming Language is the apt tool for statistical computation.

\begin{figure}[H]
    \centering
    \includegraphics[width=0.5\textwidth]{R.png}
    \caption{R sample coding in RStudio environment} 
    \label{fig:r_programming}
\end{figure}

The above figure shows the creation of a dataframe with name, age, gender, and score using R. There are 10 records; the dataframe has duplicate values for age and score.

\section{Julia}

The aim of Julia is to create a high-level programming language that is generic and clear.

\begin{figure}[H]
    \centering
    \includegraphics[width=0.5\textwidth]{J.png}
    \caption{Sample code for Julia Programming language} 
    \label{fig:julia_programming}
\end{figure}
 
The program calculates the sum of all numbers from 1 up to a specified number \( n \). This was done using the Julia Programming language.

\section{Conclusion}

This report has utilized \LaTeX\ for the documentation of the report to explore the initial stages of downloading and exploring \LaTeX, Python, R, Mathematica, and Julia. Each tool provides distinct features, contributing to effective data analysis and comprehensive documentation.

\end{document}
